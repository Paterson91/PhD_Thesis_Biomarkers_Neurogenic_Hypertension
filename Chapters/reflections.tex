\subsection{Furthering the validation of Ifit1}

This body of work has laid a strong foundation in furthering the discovery of novel biomarkers of hypertension. Using an optimised approach towards RNAseq analysis, a candidate list was produced of transcripts that were differentially expressed in the blood between the WKY and SHR strains. From this list several candidates were taken forward and validated using independent experimental approaches and additional models of hypertension. However, one of the key questions remaining unanswered is the specific aetiology of hypertension that Ifit1 expression is indicative of. It is clear that Ifit1 is abundantly expressed in a genetic model of hypertension, however when attempting to environmentally induce hypertension Ifit1 saw a counterintuitive decline in expression. It follows that Ifit1 may simply be a metric for genetic predisposition, and not one of blood pressure in its entirety. Here, only two genetic models of hypertension were assessed; the SHR and the BHR. Both of which are so closely related that it is not possible to discern whether Ifit1 upregulation is a relfection of the SHR lineage or indeed a true biomarker of genetic hypertension. Several genetically distinct models of hypertension exist, as shown in figure \ref{fig:phylogenetictree_strains}. By examining the expression of Ifit1 in these strains, it should be possible to conclude whether or not Ifit1 is indeed a metric of hypertension or simply the SHR strain itself. In order to further reconcile this uncertainty, efforts should be made to obtain the Dahl-Salt sensitive and salt resistant strains for analysis. This model of hypertension would allow for an assessment of the balance between environmentally induced and genetically predisposed forms of hypertension. 

Furthermore, due to the high-throughput nature of this study there exist many other candidate biomarkers that may well be successfully taken through the various validation strategies outlined here. Herein lies the novel benefit of RNAseq technologies; to guide biological discovery in a manner free from bias. 

\subsection{The Aetiology of Hypertension}

This study set out with the aim of identifying novel prognostic biomarkers of hypertension. Although a biomarker only needs to present an objective measure of a disease state, it would be of interest to further characterise any mechanistic involvement in hypertension, if at all. By exploring the biology of these biomarkers in the context of blood pressure regulation it may enable us to shed light on how hypertension arises. 

%What is clear, is the role of the immune system in the development of elevated blood pressure. Multiple studies have implicated pro-inflammatory molecules in the aetiology of hypertension. 

While results were inconclusive, preliminary work has been carried out to ascertain Ifit1's involvement in hypertension by characterising the leukocyte population responsible for its expression. Future research should therefore build upon this to enable a source to be identified such that a more guided literature review of leukocyte involvement in hypertension be conducted. 

\subsection{Translating the findings to a human cohort}

From these findings it is now of great interest to expand validation studies to assess how suitable these biomarkers are for a metric of human hypertension. With this in mind, the \acrfull{quod} initiative presented a novel opportunity to test biomarkers in human tissues from patients with comprehensive medical histories noted. 

The initiative was formed as part of the Nuffield Department of Surgical Sciences with the aim of identifying pathways of injury and repair in donor organs, in the hopes of mitigating any arising complications to organ recipients. It is well known that patients of donated organs from these with high blood pressure exhibit a lower survival rate \cite{Hu2017}. This present study aims to see if novel changes taking place in the blood can help to identify those that have a higher blood pressure. This would enable improved screening of postmortem donors where it is no longer possible to measure blood pressure for the biomarkers associated with chronically elevated blood pressure as identified here. Furthermore, this study is also looking towards prognostic changes that occur before the onset of high blood pressure. This would allow for treatment plans to be put in place prior to blood pressure rising and putting strain on internal organs. 

Due to the operating of the initiative, blood tissue was not available from donors. However, as the spleen is heavily vascularised it will be used as a proxy measurement of blood transcripts. At the time of writing, ethical approval was granted for usage of these tissues for biomarker validation. From an initial list of 364 potential male donors, 40 individual samples have been selected based on the clinical data provided in order to conduct a pilot study. This was carried out by narrowing down the total potential donor list to those with no history of smoking, or cardiovascular disease and binned into those with documented hypertension and those with no history of hypertension. From this narrowed list of potential candidates 20 patients were selected from either the hypertensive or normotensive cohorts based on how comprehensive their medical history was. 

Tissues have since been received from the QUOD biobank and are currently stored at -80$\degree$C ready for RNA extraction and quantification. Linear regression modelling will then enable us to assess whether any correlations between expression and blood pressure exist. This will allow the direct assessment of how suitable our biomarkers are in a human population, and to further assess how well transcripts obtained \textit{post mortem} correlate with patient blood pressure \textit{pre-mortem}. It therefore presents an interesting avenue for furthering research into biomarkers of hypertension. 

\subsection{Concluding remarks}

Given the significant health and financial impact of essential hypertension, coupled with an ever increasing population, work towards a better diagnostic approach is of huge importance. Only with continued study towards identifying the aetiology of hypertension will it result in improved diagnostic assessment, ultimately leading to a therapeutic design centred on prophylaxis rather than symptomatic management. With this in mind, High-throughput 'omics' approaches hold a great deal of promise in providing a thorough characterisation of disease states. In diagnosing patients earlier in the disease’s progression, sufficient preventative steps can be put in place to minimise the burden placed on healthcare, and ultimately an individual’s health.