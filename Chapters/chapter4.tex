\doublespacing

\section{Introduction}

The profile differences observed in Ifit1 between the \acrshort{wky} and \acrshort{shr} strains are robust and significant, however there is a possibility these relative differences flagged up by qPCR and RNAseq could be attributable to the sole comparison of the \acrshort{wky} and \acrshort{shr} transcriptomes, and therefore not useful for identifying a hypertensive strain from amongst a population of other normotensive rat strains. In order to ascertain whether these biomarker expression profiles were truly indicative of the \acrshort{shr} hypertensive strain, a blood analysis across 3 additional rat strains was conducted. The 5 total strains selected for the expression profiling were therefore; \acrfull{shr}, \acrfull{wky}, Wistar (WIS), \acrfull{sd} and \acrfull{lh}. 

The key to this selection process was to provide a spectrum of normotensive strains that may act as controls to allow for hypertension specific transcripts to be identified. For this reason, a range of genetically similar and distinct strains were selected. In addition to the \acrshort{shr} and \acrshort{wky} strains previously focused on, the more genetically distinct Wistar colony was also used from which both strains were originally obtained \cite{NAP20031}.

The Wistar rat is an outbred albino rat that represents the first breed developed for use in biological and medical research. It was originally developed at the Wistar Institute. Prior to this, laboratories used the common house mouse (\textit{Mus musculus}) as their primary model organism. More than 50\% of all laboratory rat strains that are used today are descended from the original colony, developed in 1906 \cite{Clause1998}. The Wistar strain is one of the most popular rats used for research today, and has been further used to develop many common rat stocks such as the \acrshort{shr} and \acrshort{sd}. 

One of the more interesting strains added to this experiment is that of the \acrshort{sd}. There exists a wealth of literature describing ways in which to evoke hypertension in this strain, be it surgically, pharmacologically or dietary \cite{DAngelo2005, Alexander2001, Bayorh2003}. It is therefore an appealing model to use for manipulating blood pressure levels using established methods and to study how blood pressure correlates with biomarker expression. Not only does this shed light on the correlation between blood pressure and biomarker expression, it also allows the reconciliation between the aetiology of hypertension and how transcripts are influenced. One of these methods is a chronic 3-day dehydration protocol. Previous studies have shown chronic dehydration to produce a significant increase in mean arterial pressure, from the second day of \acrfull{dh} \cite{Colombari2011}.

An extension of this genetic diversity was given by the even more genetically distinct outbred \acrshort{lh} rat strain. This was a strain originally bred from the Wistar outbred rat at the Lister Institute. Although commonly used in many experiments today the \acrshort{lh} presented a strain that has seen little involvement in hypertension research.

In order to reconcile the interface between environmentally induced hypertension and genetically predisposed hypertension the \acrfull{bhr} was later added to this series of experiments. This is a rat strain bred in order to study how a mixed genetic background can influence blood pressure. Lawler \textit{et al.}, bred female SHRs with male normotensive WKY rats to produce a F1 generation offspring that had a genetic predisposition toward elevated blood pressure. The strain only exhibits a slightly elevated blood pressure than WKY controls. However, when stressed or exposed to a diet high in salt, the strain develop overt hypertension at levels comparable to SHRs \cite{Sanders1992, Lawler1991}. For this reason, this model presents an opportunity to study the complex relationship between the genome and environment in the aetiology of hypertension. Therefore, here an investigation on the expression profiles of biomarkers in the blood for stressed and unstressed animals at both pre- and fully hypertensive ages was conducted. The stress paradigm used was one established by Šarenac \textit{et al.}, 2011 and consisted of a 9-day tube restraint course. This study replicated the stress paradigm in 4 and 12wk old animals \cite{Sarenac2011}. Using this many different control strains, from a range of different colonies, should allow for a better assessment of the suitability of the biomarker being tested.  With such a large dataset of RNAseq significant transcripts between the WKY and SHR blood, any that fail to identify the SHR from the 4 control groups can be quickly and easily omitted without further study. 

\doublespacing
\section{Aims}
The aim of this section is to take forward the work previously carried out and confirm the applicability of our previously identified biomarkers, in both genetically predisposed, control and environmentally induced models of hypertension by;\\
\begin{itemize}
\singlespacing
\setlength
	\item Assess Ifit1 expression using multiple control strains \\
	\item Use previously validated paradigms to environmentally induce acute blood pressure changes \\
	\item To profile any correlations between paradigms of inducing blood pressure changes and Ifit1 expression \\
	\item Establishing a BHR Restrained Vs Control Environmentally induced model of hypertension \\
\end{itemize}

\doublespacing

\section{Materials and Methods}

\subsection{Multi-strain Blood Analysis}

Rats from each strain (\acrshort{shr}, \acrshort{wky}, Wistar, \acrshort{sd} and \acrshort{lh}; n = 8) were ordered at either 11 weeks of age or equivalent body weight based on supplier growth curves. They were then left to habituate for 1 week before sacrifice via cranium strike and guillotine at ~12wks.Trunk blood was taken and stored along with ~2ml of RNAlater at -80$\degree$C. Brains were extracted and frozen on dry ice before storage at -80$\degree$C.  

\subsection{Sprague Dawley Dehydration}

For the dehydration group, Adult (11wks) \acrshort{sd}s were allowed to habituate for 1 week before removing water from their cages for 3 days (72 hours; onset 15:00 day 1, end 15:00 day 3). All \acrshort{dh} rats had access to food. These were then sacrificed, along with a control group of experimentally naïve \acrshort{sd}s (n=12/group), at 12wks \& 3 days of age. Trunk blood and whole brains were collected and stored at -80$\degree$C consistent with the above protocol. 

\subsection{BHR Restrained Vs Control (4wks \& 12wks)}

Here, a 9-day chronic restraint stress paradigm was implemented using the BHRs that could be compared to their age-matched Wistar controls. In addition to the stressed groups at both 4 \& 12 weeks old, age matched control groups were set aside and received no stress. Animals in groups of 7 were housed at 25$\degree$C and supplied with food \textit{ad libitum}. Juvenile and adult restraint group animals began the protocol 9 days prior to their 4th or 12th week of age respectively. The restrained protocol was the same as outlined in Šarenac \textit{et al} 2011 \cite{10.1371/journal.pone.0006162}. Briefly, each day at 9:00am, animals were habituated in the protocol room for 30mins before being placed for 60minutes in a clean plexiglass restraint ($\diameter$ 3.8cm for juveniles, 5.5cm for adults) in the supine position. After which period, the animals were weighed. At the same time, the control group of animals were weighed daily. On the ninth and final day, animals were sacrificed immediately after restraint. Trunk blood and whole brains were collected and stored at -80$\degree$C consistent with the above protocols.

\section{Results}

\subsection{Multi-strain Blood Analysis}

\begin{figure*}
\centering
\begin{tabular}{c}
  \includegraphics[width=0.7\textwidth]{chapter4/Ifit1_multistrain.pdf} \\
\end{tabular}
\caption[Multistrain Analysis of Ifit1 blood expression]{Multistrain Analysis of Ifit1 blood expression in 12-week old; \acrfull{lh}, \acrfull{sd}, \acrfull{shr}, \acrfull{wky}, and Wistar (Wis)  rats. Displayed here are $2^{-\Delta\Delta C_{T}}$ of Ifit1 normalised to $\beta$-Actin and \acrshort{wky} expression. Statistical analysis made use of one-way \acrfull{anova} with Tukey's multiple test correction. (P-Value $<$0.05, *; $<$0.01, **; $<$0.001, ***;  $<$0.0001, ****).}
\label{fig:multistrain}
\end{figure*}

Ifit1 expression analysis saw an interesting spread in blood expression across all of the various strains tested (Figure \ref{fig:multistrain}). Following a one-way \acrshort{anova}, and similar to the comparisons made in sections \ref{candidatesqpcr} and \ref{qPCR Validation of Ifit1 in New cohort of Animals}, Ifit1 saw a robust and significant increase in the \acrshort{shr} as compared with the \acrshort{wky} (Figure ~\ref{fig:multistrain}; WKY vs SHR, +26.02 Fold, $\textit{P-Value$<$0.0001}$). This highly significant increase in abundance was conserved when comparing the \acrshort{shr} with any of the other control normotensive strains (Figure ~\ref{fig:multistrain}; LH vs SHR, +26.95 Fold, $\textit{P-Value$<$0.0001}$; SD vs SHR, +22.27 Fold, $\textit{P-Value$<$0.0001}$; Wistar vs SHR, +22.87 Fold, $\textit{P-Value$<$0.0001}$;).

Despite being consistently elevated in the \acrshort{shr}, Ifit1 displayed a range of expression values within each of the multiple control strains tested. The \acrshort{lh} blood saw by far the lowest levels of Ifit1 expression, averaging at around 10\% the expression values of the \acrshort{wky}. This was a tightly clustered grouping of expression, with little deviation from the mean (Mean=0.11, \acrshort{sdev}=0.045). The \acrshort{sd} cohort saw a much greater mean and spread due to 25\% of the datapoints being greatly upregulated in expression to a level comparable with \acrshort{shr} expression (Mean=4.79; \acrshort{sdev}=8.87). Here, all expression values were normalised to \acrshort{wky} expression, as this had been the normotensive control for prior comparisons. Therefore, this became arbitrarily normalised as $2^{-\Delta\Delta C_{T}}$ values around 1.0. The spread of the \acrshort{wky} blood Ifit1 expression was relatively low compared with the other strains (Mean=1.04, \acrshort{sdev}=0.33). Finally, Wistar Ifit1 expression showed a similar spread to the \acrshort{sd} sample group, with a slightly diffuse profile of expression values, and a single outlier value on a scale comparable to those observed in the \acrshort{shr} blood expression values (Mean=4.19, \acrshort{sdev}=9.17).

\subsection{Sprague Dawley Dehydration}

\begin{figure*}
\centering
\begin{tabular}{c}
  \includegraphics[width=0.7\textwidth]{chapter4/Ifit1-DH.pdf} \\
\end{tabular}
\caption[Environmentally induced hypertension in SD rats by dehydration]{Assessment of Ifit1 expression levels in an environmentally induced model of hypertension; using 12 week old \acrfull{sd} rats and 3-days chronic dehydration (SD DH). Displayed here are $2^{-\Delta\Delta C_{T}}$ Ifit1 normalised to $\beta$-Actin and \acrshort{wky} expression. Statistical analysis made use of Student's T-Test. (P-Value$<$0.0001, ****).}
\label{fig:dehydration}
\end{figure*}

Following a 3-day chronic dehydration protocol, Ifit1 expression levels were assessed in the blood of \acrshort{sd} rats and their euhydrated controls (Figure \ref{fig:dehydration}). Similarly to within the cohort of \acrshort{sd} rats used for the multistrain analysis (Figure \ref{fig:multistrain}), Ifit1 saw a wide spread in expression values in the control euhydrated group. After 3-days of chronic dehydration, the spread of Ifit1 was much tighter than in the euhydrated cohort. Furthermore, an unpaired Student's T-Test revealed a highly significant downregulation in the dehydrated cohort as compared with the euhydrated controls (Figure ~\ref{fig:dehydration}; SD vs SD DH, -0.34 Fold, $\textit{P-Value$<$0.0001}$). 

\subsection{BHR Restrained Vs Control (4wks \& 12wks)}

\begin{figure*}
\centering
\begin{tabular}{c}
  \includegraphics[width=0.7\textwidth]{chapter4/RestraintStress.pdf} \\
\end{tabular}
\caption[Restraint stress paradigm of induced hypertension]{Restraint stress paradigm of induced hypertension, using genetically predisposed \acrfull{bhr} against their Wistar controls. Both juvenile (4-weeks) and adult (12-weeks), were subjected to a 9-day chronic restraint stress paradigm or remained as unrestrained controls. Displayed here are $2^{-\Delta\Delta C_{T}}$ of Ifit1 normalised to $\beta$-Actin and control juvenile Wistar Ifit1 expression. Statistical analysis made use of three-way \acrfull{anova} with Tukey's multiple test correction. (P-Value $<$0.05, *; $<$0.01, **; $<$0.001, ***).}
\label{fig:restraint}
\end{figure*}

Expanding upon the dehydration method of induced elevated \acrshort{bp}, a 9-day chronic restraint stress paradigm was initiated, based on Šarenac \textit{et al} \cite{Sarenac2011}. Blood expression of Ifit1 was assessed in both juvenile and adult rats (Figure \ref{fig:restraint}). Statistical analyses took the form of a three-way \acrshort{anova} and showed no comparisons to reach below the 0.05 threshold of significance. Here, the comparison of \acrshort{bhr} control \textit{versus} restrained in the juvenile cohort had the lowest P-Value of 0.08. 

\section{Discussion}

\subsection{Multi-Strain Analysis}

The Multiple control strains were key to allowing the assessment of the suitability of candidate biomarkers. Using this method, a distinction could be drawn between a hypertensive strain, whilst ensuring any expression profile differences were not traits of the \acrshort{wky} control strain alone. As this body of work is looking for a biomarker that sheds light on individual susceptibility towards hypertension, it is to be expected that a successful biomarker would be tightly grouped amongst hypertensives (\acrshort{shr}s) and spread amongst the normotensive controls. With this in mind, a genetically diverse set of control candidates were selected. 

The genetic diversity required for this analysis has been confirmed and quantified through a variety of high-throughput studies. A study in 2013 looked into the genetic spread of 27 commonly used laboratory rat strains, including a variety of models for hypertension, insulin resistance, diabetes, and their resprective control strains \cite{Atanur2013}. Using next-generation illumina sequencing Atanur \textit{et al} achieved between 10-20x read coverage, after rigorous filtering criteria, for all 27 strains. The team used the Brown Norway (BN/Mcwi) genome as a reference, and detected 9,665,340 \acrfull{snv} and 3,502,117 short indels. With this information, the team were able to calculate the distance between any pair of strains by dividing the number of differing \acrshort{snv}s by the length of the rat genome. A distance matrix could then be drawn using the Fitch-Margoliash method, with 1000 bootstraps. The resulting matrix was used to construct a phylogenetic tree (Figure \ref{fig:phylogenetictree_strains}). This independent analysis of rat strain heterogeneity illustrates the genetic diversity of the control strains selected here via the emergence of distinct clusters. 

As this continuation was primarily a way of troubleshooting the ifit1 gene as a marker of hypertension, an additional and important consideration is the role ifit1 plays in the immune response. Ifit proteins, or Interferon-Induced proteins with tetratricopeptide repeats, are a major component of the protective host defences against viral infection encoded by interferon-stimulated genes (ISGs). Ifit genes are normally expressed at very low levels of silent all together and their expression can be triggered by many stimuli, although these are mainly within the context of viral and bacterial infection \cite{Fensterl2015}. Therefore, it is important to work out whether the presence of elevated ifit1 coincides or precedes hypertension, or whether it is simply an artefact of a colony infection. For this reason, the decision was made to obtain animals from different colonies as well as different strains. 

\begin{figure*}
\centering
\begin{tabular}{c}
  \includegraphics[width=1\textwidth]{chapter4/PhylogeneticTree.jpg} \\
\end{tabular}
\caption[Phylogenetic Tree of 28 Rat Strains]{Phylogenetic Tree of 28 Rat Strains taken from Atanur \textit{et al} 2013 \cite{Atanur2013}. Over 9.6 million \acrfull{snv} were used to construct a phylogenetic tree with a scale representing genetic distance. This was calculated by dividing the number of \acrfull{snv} between a given pair of strains by the length of the reference genome (Brown Norway; BN/Mcwi). This tree was constructed using the Fitch-Margoliash method with 1,000 bootstraps. This tree illustrates the genetic diversity of the strains selected for further biomarker validation, amongst many more; \acrfull{wky}, \acrfull{shr}, \acrfull{gk}, \acrfull{mns}, \acrfull{mhs}, \acrfull{lew}, \acrfull{wag}, \acrfull{bbdp}, \acrfull{ss}, \acrfull{sr}, \acrfull{ln}, \acrfull{ll}, \acrfull{lyh}, \acrfull{f344}, \acrfull{sbh}, \acrfull{sbn}, \acrfull{fhl}, \acrfull{fhh}, \acrfull{aci}, \acrfull{le}, \acrfull{bn}.}
\label{fig:phylogenetictree_strains}
\end{figure*}

An interesting observation to note, was the separation of expression values that appeared to stratify between a lower level of expression, and a greatly elevated expression, in an almost binary manner. This was exemplified in both the \acrshort{sd} and Wistar groups, as both had outlier samples whose blood expression of Ifit1 was hugely elevated to an abundance comparable with the \acrshort{shr} levels of Ifit1 expression. This gave rise to several questions. Firstly, were these animals naturally occurring hypertensives/prehypertensives within their cohort. The strains with abundant Ifit1 expressing outliers were the Wistar and \acrshort{sd}s. Both of which represent outbred strains with a greater level of genetic variability than the \acrshort{shr} or \acrshort{wky} strains. Indeed, the \acrshort{shr} was derived from Wistars exhibiting elevated \acrshort{bp} \cite{NAP20031}.  It is conceivable that a true biomarker of hypertension, or a propensity towards hypertension, would show a natural variance in an outbred population, as this would be in line with the normal distribution of blood pressure. It was therefore hypothesised that individual biomarker expression profiles that outlie the normal range within their strain, and match the profile seen within the hypertensive strain, could be indicative of either pre- or fully hypertensive phenotypes. However, the only way to deduce this for certain would be with the addition of physiological \acrshort{bp} data alongside blood transcript expression data, to see if a correlation could be drawn between these outliers and an elevated \acrshort{bp}. Secondly, whether or not Ifit1 expression would provide a dynamic metric of blood pressure if elevated blood pressure were induced. Previous assessment of Ifit1 has been conducted in the genetically predisposed \acrshort{shr} strain, and not in an environmental model of induced hypertension. In order to answer both of these question, and in lieu of direct blood pressure measurements, a well documented environmentally-induced model of hypertension was introduced on a new cohort of \acrshort{sd}s in the form of a 3-day dehydration protocol \cite{Colombari2011,Antunes2006}.

\subsection{Sprague Dawley Dehydration}

Based on the 5-strain analysis experiments, whereby 4 out of 12 (25\%) \acrshort{sd} animals displayed elevated ifit1 expression, one might expect 25\% (n=4) of this new population to exhibit elevated ifit1. The naïve SD population saw a tight clustering of ifit1 expression values, with the majority similar to the SD population from the Multi-Strain Analysis (Figure \ref{fig:multistrain}). One of the rats showed a high expression of ifit1, comparable to both the 2 outliers of the Multi-Strain SD population, and the hypertensive SHR population. This was at a ratio lower than expected, as only one animal from a total of 12 exhibited this elevated expression profile. Interestingly, these expression profiles appear to again stratify between high and low expression, with none exhibiting expression levels in-between, as if ifit1 expression is switched on or off. 

Following the 3-day dehydration protocol, Ifit1 saw a counter-intuitive reduction in abundance. In other words, an environmentally induced elevation of blood pressure occurred alongside a significant reduction in Ifit1 expression. The reasons behind this are difficult to deduce, as a literature review of \acrshort{isg}s and dehydration revealed very little work had been conducted on this subject. Indeed, high-throughput transcriptome analysis on male \acrshort{sd}s from our own group has revealed a reduction in the total number of transcripts detected following a 3-day dehydration protocol \cite{Qiu2007,Hazell2012}. However, these studies were all conducted in central regions of the brain tasked with cardiovascular and osmolarity control (\acrfull{son}; \acrfull{pvn}; and \acrfull{nil}) and not within whole blood. This reduction in abundance suggests Ifit1's inability to act as a dynamic metric of blood pressure when it is environmentally manipulated, however it does not belie its potential utility as a biomarker of genetically predisposition towards hypertension. One way of confirming this hypothesis further would be an analysis of its expression in the Dahl-Salt Sensitive model of hypertension. 

The Dahl-Salt sensitive model of hypertension represents a strain with a genetic predisposition towards hypertension. The strain can be split into two, based on their responsiveness towards a high salt diet; Salt sensitive and Salt resistant rats. Both of which steadily become hypertensive given a diet of standard rat chow. However, once given a diet high in salt, the salt sensitive cohort will develop a severe and fatal form of hypertension. As this model marries genetic predisposition (as seen in the \acrshort{shr}) with environmentally induced hypertension (as seen in the \acrshort{sd} dehydration study), it may allow for the assessment of Ifit1 as a metric of the genome-environment aetiology of hypertension.
   
\subsection{\acrshort{bhr} Restraint}

As the Dahl-Salt sensitive model was not available here, this work could be furthered in a similarly predisposed model by use of the \acrfull{bhr}. Due to its close relation to the \acrshort{shr} model, the \acrshort{bhr} could be used to assess ifit1 expression in a similarly genetically predisposed strain, with only a modest elevated \acrshort{bp} and in \acrshort{bhr}s subjected to an environmentally induced form of full hypertension. 

The first thing noted by this study, was the high level of variability in the juvenile animals, regardless of strain or restraint. This high level of variation prevented any statistically significant calls being made by the three-way \acrshort{anova} employed here. Within the \acrshort{bhr} group, the level of Ifit1 expression was more or less comparable to that observed within the Wistar controls. This is unlike the hugely robust fold changes observed within the \acrshort{shr} animals as compared with their normotensive controls. Indeed, there does appear a slight increase in expression in the \acrshort{bhr} animals as compared with the Wistars, although this is not significant. A similar non-significant pattern is observed between the adult cohorts.

The application of restraint stress appears to do little in modulating blood expression of Ifit1, regardless of the age group or strain tested. The juvenile \acrshort{bhr} control \textit{versus} restraint comparison has the lowest P-Value (P-Value = 0.08), showing an almost significant reduction of expression. 

Here, the expression of Ifit1 was not significantly modulated in either control or restraint-stress treated animals. This was despite a model of induced hypertension within a genetically predisposed strain, that has been well documented to show an elevated \acrshort{bp} \cite{Sarenac2011}. Taken together with the results of the \acrshort{sd} dehydration study, this is perhaps indicative of Ifit1's inability to serve as a biomarker of borderline or environmentally induced hypertension. In order to confirm this conclusion, the Dahl salt sensitive model would prove invaluable.

However, due to the genetic similarity between the \acrshort{bhr} and \acrshort{shr} animals, it is promising to note a relatively low expression value in the control \acrshort{bhr}. As this suggests the model destined to get hypertension (\acrshort{shr}) alone sees this over abundance of the transcript, and not the model of borderline hypertension.

\subsection{Concluding remarks}

This chapter was broadly successful in further characterising the ability of Ifit1 to discern between a hypertensive strain, and a cohort of normotensive controls. However, unlike the \acrshort{wky} \textit{versus} \acrshort{shr} comparison which could be easily resolved by the stratification of highly or lowly expressed Ifit1, two of the control strains also saw elevated Ifit1 expression in the blood. While it was not possible to correlate elevated Ifit1 expression with an induced form of elevated blood pressure, Ifit1 may still serve as a suitable biomarker for assessing the distribution of blood pressure across outbred strains. 

Since much of this work has been conducted via proxy animals following validated methods of inducing hypertension, future work should make use of direct physiological measurements in order to confirm a rise in blood pressure has truly taken place. By using plotting blood pressure changes, along with monitoring expression levels of Ifit1, a correlation can be drawn between the two. This way, Ifit1 suitability can be measured as a proxy metric of blood pressure variation, regardless of strain. 



%\subsection{Multi-Strain Analysis - BHR Restraint Combined}

%Previous data from the BHR restrain vs control experiment was revisited as the ages used for the adult BHR and their control Wistar adult Rats were 12 weeks, and therefore comparable to the rats used for the Multistrain blood analysis experiment. Furthermore, blood was obtained and processed in exactly the same way for both experiments. Once qPCR Ct threshold values were standardised for our housekeeper ($\beta$-Actin) and our target gene for both experimental groups, $\Delta$Ct values could then be directly compared between the two. The inclusion of the Wistar strain in both experiments was useful in assessing comparability as expression levels of ifit1 were similarly spread, despite being obtained from geographically separate colonies at different periods in time. 

%Observing all $\Delta$Ct values together, we can see an interesting trend emerging. Although the groups for the restraint tests appear slightly more dispersed than the Multistrain analysis experiment, they seem to be stratifying into the higher and lower expression bins that are seen in the Multistrain analysis experiment. The control Wistars from both groups do indeed show similar spreads in the expression of ifit1. The BHR controls show a much higher and tighter expression profile of ifit1, and one comparable to the SHRs from the other experiment. Considering how BHRs are obtained from the breeding of one SHR and one Wistar, this similar expression is to be expected. It is interesting to note that the application of chronic restraint stress in this protocol, while sufficient to elicit the hallmarks of pre-hypertension in BHRs, does not appear to influence the expression of ifit1 for either the Wistar or BHR strains. Perhaps this indicates ifit1 as being unsuitable for correlating with environmental forms of hypertension, instead being a candidate for genetic predisposition.


